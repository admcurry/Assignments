\documentclass{article}
\usepackage[utf8]{inputenc}
\usepackage{geometry}
\newgeometry{left=2cm,bottom=2cm, right=2cm, top=2cm}
\usepackage{graphicx}
\usepackage{url}
\usepackage{parskip}
%--------------------------------------------------------------------------------
\title{Problem Sheet 1}
%--------------------------------------------------------------------------------
\begin{document}
Probability \& Statistics Problem Sheets\\
Adam Curry
\vspace{3 mm}
\hline
\hspace{25 mm}

%--------------------------------------------------------------------------------
\large{\noindent{\textbf{Problem Sheet 1}}}\\
%--------------------------------------------------------------------------------

\begin{figure}[h]
\centering
\textbf{Picking red balls from a bag containing 10 red balls and 20 blue balls:}\\
\includegraphics[scale=0.35]{Rplot1.png}
\includegraphics[scale=0.35]{Rplot2.png}
\includegraphics[scale=0.4]{Rplot3.png}

The hypergeometric distribution has the largest probabiltiy, and the binomial distrubutions probabilities are closer together - but they overall have the same shape.'
\centering
\end{figure}

\begin{figure}[h]
\centering
\textbf{Picking red balls from a bag containing 1000 red balls and 2000 blue balls:}\\
\includegraphics[scale=0.35]{Rplot4.png}
\includegraphics[scale=0.35]{Rplot5.png}
\includegraphics[scale=0.4]{Rplot6.png}

In the second senario with 1000 red and 2000 blue, the hypergeometric distribution matches the binomial distribution almost exactly.
\centering
\end{figure}


\end{document}
