\documentclass{article}
\usepackage[utf8]{inputenc}
\usepackage{geometry}
\newgeometry{left=2cm,bottom=2cm, right=2cm, top=2cm}
\usepackage{graphicx}
\usepackage{url}
\usepackage{parskip}
\usepackage{listings}
\usepackage{amssymb}
%--------------------------------------------------------------------------------
\title{Problem Sheet 3}
%--------------------------------------------------------------------------------
\begin{document}
Probability \& Statistics Problem Sheets\\
Adam Curry
\vspace{3 mm}
\hline
\hspace{25 mm}

%--------------------------------------------------------------------------------
\begin{center}
    \large{\noindent{\textbf{Problem Sheet 3}}}\\
\end{center}
%--------------------------------------------------------------------------------
\section*{Problem 1}

You work for a hedge fund which employs a number of servers to perform critical
statistical arbitrage calculations. The probability that a server fails on a given day is 0.5\% (p = 0.005) and our hardware insurance covers the first three repairs free of cost, after that we must pay for repairs.\\
Calculate the probability that a server lasts for a full 365-day year (or more) without incurring any hardware repair expenses.\\

\textbf{Answer:}
\begin{lstlisting}[language=R]
noexpense <- 1 - pnbinom(365-4, 4, 0.005)
noexpense
\end{lstlisting}

Probability = 0.8877234

%--------------------------------------------------------------------------------
\section*{Problem 2}

On average we get 5.6 Hammersmith \& City trains through Stepney Green station
every hour. \\
Assuming the number of trains in one hour follows a Poisson distribution, calculate the probability that we get no trains in one hour.

\textbf{Answer:}
\begin{lstlisting}[language=R]
notrains <- dpois(0, 5.6)
notrains
\end{lstlisting}

Probability = 0.003697864

%--------------------------------------------------------------------------------
\section*{Problem 3}

Let X be the outcome of rolling a die.\\
Calculate the moment generating function of the random variable $X$, $M_X(t)$.\\
Verify that $M_X(0) = \mathbb{E}(X^0)$ and $M′_X(0) = \mathbb{E}(X^1)$.

*to complete*

\end{document}
